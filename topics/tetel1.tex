\chapter{Implicit- és inverzfüggvény}\label{chap:tetel1}

\section{Követelmények}
\begin{enumerate}
    \item Implicitfüggvény fogalma
    \item Kapcsolata a feltételes szélsőérték problémával és inverzfüggvénnyel
    \item Implicitfüggvény-tétel
    \item Inverzfüggvény-tétel
    \item Inverzfüggvény-tétel bizonyításának vázlata
\end{enumerate}

\section{Bevezető példa}

Az implicitfüggvények megértéséhez hasznos először az előadáson is látott példát venni. Ennek egy részletesebb kidolgozása megtalálható a 2. tétel elején (ld. \ref{sec:t2-bev} szekció), bár ott inkább a feltételes szélsőértékekre kerül a hangsúly.

Feladat: az egységnyi kerületű téglalapok közül keressük a legnagyobb területűt.

Jelöljék ehhez az oldalakat $x,y \in \R$, valamint a területet az $f(x,y) := x \cdot y \quad ((x,y) \in \R^2)$ függvény. Az egységnyi kerület feltételét pedig fogalmazzuk meg a következő módon: vegyük a $g(x,y) := 2x + 2y - 1 \quad ((x,y) \in \R^2)$ függvényt, ekkor az $x,y$ oldalú téglalap pontosan akkor egységnyi kerületű, ha $g(x,y) = 0$.

A fentiek alapján az $\restrict{f}{\{g = 0\}}$ leszűkítés (abszolút) maximumát keressük, ahol
$$\{g = 0\} := \{(x,y) \in \R^2 \mid g(x,y) = 0\}$$

Mivel erre a leszűkítésre nem alkalmazhatóak a szokásos szélsőérték tételek (ld. \chapref{chap:tetel2}), ezért egyéb megoldást keresünk. Látszik, hogy $y$ itt könnyen kifejezhető $x$-ből, ugyanis
$$g(x,y) = 2x + 2y - 1 = 0 \implies y = \frac 12 - x$$

Vezessük be a $h(x) := \dfrac 12 - x \enspace ( x\in \R)$ függvényt, ez tehát azt írja le tulajdonképpen, hogy milyen összefüggés van a két változó között. Ezzel kapunk egy új felírást $\{g=0\}$-ra:
$$\{g = 0\} = \graf h = \{(x, h(x)) \mid x \in \R\} \subset \R^2$$

Tehát $(x, h(x)) \in \R^2$ mindig egy egységnyi kerületű téglalapot leíró számpáros lesz.

Ekkor az $\restrict f {\{g=0\}}$ leszűkítés maximumának keresése megegyezik a következő $\Phi : \R \to \R$ függvény maximumának keresésével:
$$\Phi(x) := f(x,h(x)) = x \cdot \left(\frac 12 - x\right) \quad (x \in \R)$$

Mivel ekkor már $\dom_\Phi = \R$, ezért bármely $x \in \R$-beli pontra $x \in \intp \dom_\Phi$, vagyis vizsgálhatjuk a függvényt a ``szokásos'' lokális szélsőérték tételeink segítségével.\footnote{Mivel $\Phi \in D^2$ is teljesül.}

\begin{gather*}
\Phi'(x) = 1 \cdot \left(\frac 12 - x\right) + x \cdot \left(0 - 1\right) = \frac 12 - 2x \quad (x \in \R) \\
\Phi''(x) = -2 < 0\quad (x \in \R)
\end{gather*}

És mivel,
$$\Phi'(x) = \frac 12 - 2x = 0 \eqv x = \frac 14$$
ezért $\Phi$-nek $x = \dfrac 14$-ben maximuma van. Az ehhez tartozó $y$-t könnyen kapjuk $h$-n keresztül:
$$y = h\left(\dfrac 14\right) = \frac 12 - \frac 14 = \frac 14$$

Vagyis az egységnyi kerületű téglalapok közül a legnagyobb területű oldalainak a hossza:
$$x = y = \dfrac 14$$

\section{Implicitfüggvény}

A fent látott változók közötti ``függőséget'' próbáljuk általánosítani.

Legyenek $n,m \in \N$, amikre $2 \le n, 1 \le m < n$ teljesülnek. Alkalmazzuk $\R^n$ felett a következő felbontást:
$$\xi = (\xi_1, \ldots, \xi_n) \in \R^n \ \longrightarrow\ x := (\xi_1,\ldots,\xi_{n-m}) \in \R^{n-m},\ y := (\xi_{n-m+1},\ldots, \xi_n)\in \R^{m}$$
Jelölje $\xi = (x,y)$. Ez röviden az $\R^n \equiv \R^{n-m} \times \R^m$ megfeleltetéssel egyenértékű.

\begin{definition}[implicitfüggvény]
Legyen $f = (f_1, \ldots, f_m) \in \R^n \to \R^m$, amire a fenti megfeleltetés alapján $f \in \R^{n-m} \times \R^m \to \R^m$.
Tegyük fel, hogy egy $(a,b) \in \dom_f$ helyre $f(a,b) = 0$.

Tegyük fel továbbá, hogy $\exists K(a) \subset \R^{n-m}$ és $\exists K(b) \subset \R^m$, amikkel bármely $K(a)$-beli $x$-hez egyértelműen tartozik $K(b)$-beli $y$, ahol $f(x,y) = 0$. Formálisan:
$$\exists K(a)\subset \R^{n-m},\,K(b)\subset \R^m : \forall x \in K(a) : \exists!\, y \in K(b) : f(x,y) = 0$$

A fenti feltételek mellett megadható a
$$\varphi : K(a) \to K(b),\ \varphi(x) := y$$
függvény.

Ekkor azt mondjuk, hogy $\varphi$ az $f$ által (az $(a,b)$ körül) meghatározott \textbf{implicitfüggvény}.
\end{definition}

\subsection{Implicitfüggvény tulajdonságai}

Vizsgáljuk meg az implicitfüggvényt, valamint a kapcsolatát $f$-el, megtartva a definícióban használt jelöléseket.

Egyértelmű a definíció alapján, hogy $\forall x \in K(a) : f(x, \varphi(x)) = 0$. Azt is tudjuk, hogy $y := \varphi(x)$ az egyetlen olyan $K(b)$-beli pont, amire $f(x,y) = 0$. Nyilvánvaló, hogy $\varphi(a) = b$.

Legyen $F(x,y) := (x,f(x,y)) \enspace ((x,y) \in \dom_f)$, ekkor $F \in \R^n \to \R^n$ és $F(a,b) = (a,0)$.

Az összetett függvény deriváltja alapján, ha egy $(x,y) \in \dom_f$ pontban $f \in D\{(x,y)\}$, akkor $F \in D\{(x,y)\}$ és
$$F'(x,y) =
\begin{bmatrix}
I & \Theta \\
\partial_1f(x,y) & \partial_2f(x,y)
\end{bmatrix}
\in \R^{n \times n}$$

Ahol $I$ az $\R^{(n-m)\times(n-m)}$-beli egységmátrix, $\Theta$ pedig az $\R^{(n-m)\times m}$-beli nullmátrix.

Szerintem a kibontott verzióban kicsit könnyebb megérteni a mátrix szerkezetét:
\begin{gather*}
F'(x,y) = \\
\begin{bmatrix}
    \partial_{x_1}x_1 =1 & \partial_{x_2}x_1 =0 & \cdots & \partial_{x_{n-m}}x_1 = 0 & \partial_{y_1}x_1 =0 & \cdots & \partial_{y_m}x_1 = 0 \\
    \partial_{x_1}x_2 = 0 & \partial_{x_2}x_2 = 1 & \cdots & \partial_{x_{n-m}}x_2 = 0 & \partial_{y_1}x_2 = 0 & \cdots & \partial_{y_m}x_2 = 0 \\
    \vdots  &   & \ddots & \vdots & \vdots &  \ddots & \vdots  \\
    \partial_{x_1}x_{n-m} = 0 & \partial_{x_2}x_{n-m} = 0 & \cdots & \partial_{x_{n-m}}x_{n-m} = 1 & \partial_{y_1}x_{n-m} = 0 & \cdots & \partial_{y_m}x_{n-m} = 0 \\
    \\
    \partial_{x_1}f_1(x,y) & \partial_{x_2}f_1(x,y) & \cdots & \partial_{x_{n-m}}f_1(x,y) & \partial_{y_1}f_1(x,y) & \cdots & \partial_{y_m}f_1(x,y) \\
    \partial_{x_1}f_2(x,y) & \partial_{x_2}f_2(x,y) & \cdots & \partial_{x_{n-m}}f_2(x,y) & \partial_{y_1}f_2(x,y) & \cdots & \partial_{y_m}f_2(x,y) \\
    && \ddots & &&& \ddots\\
    \partial_{x_1}f_m(x,y) & \partial_{x_2}f_m(x,y) & \cdots & \partial_{x_{n-m}}f_m(x,y) & \partial_{y_1}f_m(x,y) & \cdots & \partial_{y_m}f_m(x,y) \\
\end{bmatrix}
\end{gather*}

Persze vigyázni kell az informális jelölésekkel (pl. $\partial_{x_1}x_1$).

Itt tehát $\partial_1 f(x,y) \in \R^{m\times(n-m)}$ és $\partial_2f(x,y) \in \R^{m\times m}$.

Továbbá ha $f \in C^1\{(a,b)\}$, akkor $F \in C^1\{(a,b)\}$

Az egységmátrix ``mentén'' számolva könnyen megkapható, hogy
$\det F'(a,b) = \det \partial_2f(a,b)$, tehát
$$\det \partial_2f(a,b) \ne 0 \implies \det F'(a,b) \ne 0$$

Ez az összefüggés először kicsit légből kapottnak tűnhet, de (többek között) erre fog alapulni az inverzfüggvény-tétel és az implicitfüggvény-tételek közötti kapcsolat.

\begin{theorem}[implicitfüggvény-tétel]
Adott $n,m \in \N,\ 2 \le n,\ 1 \le m < n$ mellett egy
$$f \in \R^{n-m} \times \R^m \to \R^m$$
függvény. Tegyük fel, hogy $f \in C^1$ és
$$\exists(a,b) \in \intp \dom_f : f(a,b) = 0 \ \textnormal{ és } \det \partial_2f(a,b) \ne 0$$

Ekkor alkalmas $K(a),\, K(b)$ mellett létezik $f$ által $(a,b)$ körül meghatározott $\varphi : K(a) \to K(b)$ implicitfüggvény.
Továbbá $\varphi \in C^1$ és
$$\varphi'(x) = -\partial_2f(x,\varphi(x))^{-1} \cdot \partial_1 f(x,\varphi(x)) \quad (x \in K(a))$$
\end{theorem}

\newpage

\section{Lokális invertálhatóság}

\subsection{Emlékeztető}

\begin{theorem}[egyváltozós inverzfüggvény-tétel]
    Legyen $f \in \R \to \R, \ f \in C^1\{a\}$ valamilyen $a \in \intp \dom_f$ pontban.

    Ha $f'(a) \ne 0$ és $\exists r >0$, amire $I := (a-r, a +r) \subset \dom_f$, akkor létezik a $g :=\left(\restrict f I\right)^{-1}$ inverzfüggvény, továbbá $g \in D$ és
    $$g'(x) = \frac 1 {f'(g(x))} \quad (x \in \dom_g)$$
\end{theorem}

\subsection{Többváltozós eset}

\begin{definition}[lokális inverz]
    Legyen adott $1 \le n \in \N$ és $f \in \R^n \to \R^n$ egy $a \in \intp \dom_f$ ponttal. Az $f$ függvény \textbf{lokálisan invertálható} $a$-ban, ha $\exists K(a) \subset \dom_f$, ahol a $g := \restrict f {K(a)}$ leszűkítés invertálható. Ekkor $g^{-1}$ inverzfüggvény az $f$ $a$-beli \textbf{lokális inverze}.
\end{definition}

\begin{theorem}[inverzfüggvény-tétel]
    Legyen $1 \le n \in \N$ mellett $f \in \R^n \to \R^n$, amire $f \in C^1\{a\}$ egy $a \in \intp \dom_f$ pontban. Tegyük fel továbbá, hogy $f$ $a$-beli Jacobi-mátrixa invertálható, azaz $\det f'(a) \ne 0$.

    Ekkor $f$ az $a$-ban lokálisan invertálható, tehát egy alkalmas $K(a) \subset \dom_f$ mellett $\restrict f {K(a)}$ leszűkítés invertálható. Továbbá a $h := \left(\restrict f {K(a)}\right)^{-1}$ lokális inverzfüggvény folytonosan differenciálható, és
    $$h'(x) = \left(f'(h(x))\right)^{-1} \quad (x \in \dom_h)$$
\end{theorem}

\begin{proof}[Bizonyítás (vázlat)]
    Az inverzfüggvény-tétel bizonyításából ennyit fogunk belátni formálisan, vagyis csak a lokális inverz létezését és folytonosságát. A bizonyítás után szerepelni fog a teljes inverzfüggvény-tétel, valamint vázlatosan a bizonyítás hátralévő lépései.

Definiáljuk először a \emph{zárt környezeteket:} $\overline{K_r(\alpha)} := \{z \in \R^n : \norm {z - \alpha}_\infty \le r\} \enspace (r > 0, \alpha \in \R^n)$.

Megjegyezzük, hogy így spec. $z \in \overline{K_r(0)} \eqv \norm z_\infty \le r$.
Jelölje $b := f(a)$.

Ekkor $a = b = 0$ és $f'(0) = I \in \R^{n \times n}$ feltehető, mivel a $G(x) := (f'(a))^{-1}\cdot (f(x+a) - f(a))$ függvény ugyanúgy viselkedik $0$-ban vizsgálva, mint $f$ az $a$-ban\footnote{Levezetéshez ld. \ref{demo:inv-trans}, bár }.

\emph{Ötlet:} Fejezzük ki $x$-et az $f(x) = y$ egyenlőségből:
\begin{equation}\label{eq:loc-inv-starter}
f(x) = y \eqv y - f(x) = 0 \eqv y + \underbrace{x - f(x)}_{=:\,h(x)} = x \eqv y + h(x) = x
\end{equation}

Itt $h \in C^1$, mivel $f \in C^1$.

Rögzítsünk egy $y \in \R^n$ vektort, válasszuk $r > 0$-t úgy, hogy $\overline{K_r(0)} \subset \dom_f$, és definiáljuk továbbá a $F_y(x) := y + h (x) \quad (x \in \overline{K_r(0)})$ függvényt.\footnote{
Vegyük észre, hogy a \eqref{eq:loc-inv-starter} ekvivalencia alapján egy tetszőleges $y \in \R^n$ vektorra pontosan akkor fog teljesülni $f(x) = y$, ha $F_y(x) = x$. Innen jön az ötlet, hogy induljunk el az $F_y(x)$-en vett fixpont-tétel felé, ami majd megadja ezt az $x$-et.
}

Ekkor a feltevésekből következően
$$h'(0) = \text{id}'(0) - f'(0) = I - I = (0)_{i,j=1}^n =: \Theta \in \R^{n \times n},$$
azaz az $n \times n$-es nullmátrix.

Feltehető továbbá, hogy $\forall x \in \overline{K_r(0)} : \norm {h'(x)}_\infty \le \dfrac 12$.

Ezt megtehetjük, mivel $h \in C^1$ és $h'(0) = \Theta$ garantálják, hogy lesz $0$-nak környezete, ahol ``elég kicsit'' mozdultak csak el ehhez a parciális deriváltak. A korábbi $\overline{K_r(0)} \subset \dom_f$ kikötésünk pedig ezzel összeegyeztethető, mivel $0 \in \intp \dom_f$.

Ebből következően\footnote{
A $(*)$-al jelölt lépéséhez a Lagrange-féle középérték-egyenlőtlenséget alkalmazzuk. \\
\href{https://en.wikipedia.org/wiki/Mean_value_theorem\#Mean_value_inequality}{Wikipédia: Mean Value Theorem}
}
\begin{gather}
\forall x,z \in \overline{K_r(0)} : \norm {F_y(x)-F_y(z)}_\infty = \norm {y - h(x) - y + h(z)}_\infty = \notag \\
= \norm {h(x)-h(z)}_\infty \stackrel{(*)}\le \sup\{\norm{h'(\xi)}_\infty : \xi \in \overline{K_r(0)}\} \cdot \norm{x - z}_\infty
\le \dfrac 12 \cdot \norm{x-z}_\infty \label{eq:loc-inv-F-approx}
\end{gather}

Becsüljük $F_y(x)$ függvényérték $\infty$-norma szerint vett hosszát:
\begin{gather*}
\norm{F_y(x)}_\infty = \norm{y + h(x)}_\infty = \lVert{y + h(x) - \underbrace{h(0)}_{=\,0}}\rVert_\infty \le \norm y_\infty + \norm{h(x)-h(0)}_\infty \\
\implies \text{ld. }\eqref{eq:loc-inv-F-approx} \implies \norm{F_y(x)}_\infty \le \norm{y}_\infty + \frac 12\norm{x}_\infty \le \norm y_\infty + \frac r2
\end{gather*}

Tehát $y \in K_{r/2}(0)$ választással,
\begin{equation}\label{eq:loc-inv-Fy-rng}
\norm{F_y(x)}_\infty < r \quad \text{és így} \quad F_y : \overline{K_r(0)} \to K_r(0).
\end{equation}

Vegyük a $\varrho(x,y) := \norm{x -y}_\infty ( = \varrho_\infty(x,y))$ metrikát. Ekkor a $(\overline{K_r(0)},\varrho)$ metrikus tér teljes\footnote{
Ld. \emph{Analízis III., $p$-normák teljessége zárt intervallumon}. Ehhez a megállapításhoz volt fontos, hogy a környezetünk zárt legyen, ugyanis $(K_r(0),\,\varrho_\infty)$ metrikus tér nem lenne teljes.}\,\footnote{
Itt az, hogy a norma által indukált metrikával kapott metrikus térre kimondtuk hogy teljes, ekvivalens azzal, hogy a $(\overline{K_r(0)}, \norm{.}_\infty)$ normált tér egy \emph{Banach-tér}.}.

A \eqref{eq:loc-inv-F-approx} egyenlőtlenségben kapott becsléssel azt is beláttuk, hogy $F_y$ egy kontrakció, $q = 0.5$ kontrakciós együtthatóval. Továbbá $K_r(0) \subset \overline{K_r(0)}$ és a \eqref{eq:loc-inv-Fy-rng}-ben tett megállapításunk alapján alkalmazhatjuk $F_y$-ra a \emph{fixpont-tételt}.

Vagyis
\begin{equation*}
    \forall y \in K_{r/2}(0) : \exists! x_y \in \overline{K_r(0)} : F_y(x_y) = x_y
\end{equation*}

Ez alapján definiáljunk egy $\varphi : K_{r/2}(0) \to \overline{K_r(0)}$ leképezést, amire $\varphi(y) := x_y \enspace (y \in K_{r/2}(0))$.
Erre minden $y \in K_{r/2}(0)$ mellett teljesül, hogy $F_y(\varphi(y)) = \varphi(y)$, tehát
\begin{equation*}
    F_y(\varphi(y)) = y + \varphi(y) - f(\varphi(y)) = y + h(\varphi(y)) = \varphi(y)
\end{equation*}

Továbbá a $h(x) = x - f(x)$ definíció alapján,
\begin{equation*}
    y + \varphi(y) - f(\varphi(y)) = \varphi(y) \implies f(\varphi(y)) = y
\end{equation*}

Felhasználva a \eqref{eq:loc-inv-Fy-rng} korlátunkat, megállapítjuk, hogy
\begin{equation*}
    \norm{\varphi(y)}_\infty = \norm{F_y(\varphi(y))}_\infty < r \quad \text{és így} \quad \varphi : K_{r/2}(0) \to K_r(0)
\end{equation*}

Belátható továbbá\footnote{
    Ezt az előadás és a segédanyag se részletezte. Nem feltétlen bonyolult, kihozható például a középérték-egyenlőtlenségből és a bi-Lipschitz tulajdonságból. Viszont ez megint új sugarak, korlátok és tételek behozását igényelné, ezért itt se bonyolítom a bizonyítást vele. Egy kicsit bonyolultabb módszer megtalálható a könyvben is.\\
    \href{https://en.wikipedia.org/wiki/Lipschitz_continuity\#Definitions}{Wikipédia: Bi-Lipschitz tulajdonság (a \emph{Definitions} szekció végén található)} % TODO: bevanal2-ből kiírni függelékbe
}, hogy $\rng_\varphi$ nyílt, és valamilyen $S := K(0) \subset \rng_\varphi \subset \overline{K_r(0)}$ környezettel véve az $\restrict f S$ leszűkítés injektív. Tehát $f$ lokálisan invertálható a $0$ pontban és a lokális inverze a $\varphi$ függvény $\rng_{\restrict f S}$ halmazra vett leszűkítése lesz. \partqed

{ % group introduced for \ty = \tilde y shortcut
\newcommand{\ty}{\tilde y}

Következő lépésnek vizsgáljuk $\varphi$ folytonosságát. Legyenek $y, \ty \in K_{r/2}(0)$, ekkor
\begin{gather*}
    \norm{\varphi(y) - \varphi(\ty)}_\infty = \norm{x_y - x_{\ty}}_\infty = \norm{F_y(x_y) - F_{\ty}(x_{\ty})}_\infty = \\
    = \norm{y + h(\varphi(y)) - \ty - h(\varphi(\ty))}_\infty \le \norm{y - \ty}_\infty + \norm{h(\varphi(y)) - h(\varphi(\ty))}_\infty
\end{gather*}

Továbbá mivel $h$ egy kontrakció (ld. \eqref{eq:loc-inv-F-approx} korlát),
\begin{gather*}
    \norm{\varphi(y)-\varphi(\ty)}_\infty \le
    \norm{y - \ty}_\infty + \norm{h(\varphi(y)) - h(\varphi(\ty))}_\infty \le \norm{y - \ty}_\infty + \frac 12 \norm{\varphi(y) - \varphi(\ty)}_\infty \\
    \implies \norm{\varphi(y) - \varphi(\ty)}_\infty \le 2 \norm{y - \ty}_\infty 
\end{gather*}

Tehát teljesül $\varphi \in C\big(K_{r/2}(0)\big)$ is. A korábbi leszűkítést nézve mivel $\rng_{\restrict f S} \subset K_{r/2}(0)$, ezért spec. a $\left(\restrict f S \right)^{-1} = \restrict \varphi {\rng_{\restrict f S}}$ lokális inverz folytonos lesz az egész értelmezési tartományán. \partqed
}

Vizsgáljuk meg végül a lokális inverzet differenciálás szempontjából.

Vezessük be a $g$ függvényt a következő módon:
$$g \in \R^n \times \R^n \to \R^n : g(u,v) := f(v) - u \quad (u \in \R^n,\, v \in \dom_f)$$

Ekkor $g \in C^1$ (mivel $f \in C^1$), $g(b,a) = f(a) - b = 0$. A deriváltjára pedig
$$\grad g(u,v) = \begin{bmatrix}-I & f'(v)\end{bmatrix}$$

Vagyis spec. $\partial_2 g(b,a) = f'(a) = f'(0) = I$ nem szinguláris. Mivel ezek a feltételek mind teljesülnek, alkalmazhatjuk az \emph{implicitfüggvény-tételt}.\footnote{Ehhez persze fel kell tenni, hogy már bebizonyítottuk a tételt korábban. A következő szekcióban kifejtésre kerül a két tétel bizonyítása közti szoros kapcsolat.}

Tehát $\exists \psi : U \to V$ függvény, ami $g(u,v)$ által a $(0,0)$ körül meghatározott implicitfüggvény, azaz $\forall u \in U : g(u,\, \psi(u)) = 0$ (ahol $U, V$ alkalmas sugarú környezetei $0$-nak).

Továbbá $\psi \in C^1$ és $\psi'(u) = -\partial_2g(u,\, \psi(u))^{-1} \cdot \partial_1 g(u,\, \psi(u)) \enspace (u \in U)$.

Ebből nekünk az utolsó tulajdonság lesz hasznos, mivel
\begin{equation*}
    \psi'(x) = -\partial_2g(x,\, \psi(x))^{-1} \cdot \partial_1 g(x,\, \psi(x)) = -\big(f'(\psi(x))\big)^{-1} \cdot -I = \big(f'(\psi(x))\big)^{-1} \quad (x \in U)
\end{equation*}

Ha behelyettesítjük $\psi$-t a $g$ függvénybe, akkor $g(x,\,\psi(x)) = f(\psi(x)) - x = 0 \enspace (x \in U)$. Tehát minden $x \in U$ esetén $f(\psi(x)) = x$. Továbbá minden $x \in U$ értékhez a $\psi(x) =: y \in V$ az egyetlen olyan $y$, amivel $g(x,y) = 0$.

A korábbiakból tudjuk, hogy $f$-nek létezik lokális inverze a $0$-ban, ez a $\varphi$ függvény. Tudjuk továbbá, hogy $\dom_\varphi$ nyílt, vagyis $U \subset \dom_\varphi$ feltehető.

Mivel $f(\varphi(x)) = x$ is teljesül minden $x \in U$-re, ezért $\psi(x) = \varphi(x) \enspace (x \in U)$.

Továbbá mivel tudjuk, hogy $\psi \in C^1$ és $U \subset \dom_{\varphi}$, ezért
\begin{equation*}
    \forall x \in U : \psi \in D\{x\} \implies \varphi \in D\{x\} \text{ és } \varphi'(x) = \psi'(x) = \big(f'(\varphi(x))\big)^{-1}
\end{equation*}

Tehát azt is beláttuk, hogy $\varphi \in C^1$ és $\varphi'(x) = \big(f'(\varphi(x))\big)^{-1}$.
\end{proof}

\subsection{Inverz- és implicitfüggvény-tétel kapcsolata}

Az \emph{ImpFT} (implicitfüggvény-tétel) és az \emph{InvFT} (inverzfüggvény-tétel)  bizonyítása nagyon szoros kapcsolatban áll egymással. Az egyik oldalt láttuk is, amikor az InvFT bizonyításának utolsó részéhez felhasználtuk az ImpFT-t. Ami ezekből viszont nem látszik, hogy a másik irányba is egy hasonlóan szoros kapcsolat található.

Felmerülhet például a kérdés, hogy nem lett-e volna könnyebb az InvFT bizonyításban, ha már sokkal korábban behozzuk a $g$-ből kapott $\psi$ implicitfüggvényt (amiről később belátjuk, hogy a lokális inverzzel megegyezik), és az első két állítást is ennek segítségével látjuk be. A rövid válasz az, hogy \emph{de, ugyanis az implicitfüggvény-tétel segítségével minden fenti tulajdonság (kb.) triviálisan megadható.} Valamint azt se lenne túl bonyolult belátni, hogy a kapott implicitfüggvény a lokális inverz kritériumainak megfelel (mi is megtettük a bizonyítás során).

A probléma azzal van, hogy az ImpFT bizonyításához szükségünk van az InvFT egy gyengébb verziójára, mint lemma. Ez a lemma az, amit az első két részében beláttunk a bizonyításnak (szándékosan az ImpFT használata nélkül). Ez kimondja a lokális invertálhatóságot, a lokális inverz létezését, és annak folytonosságát (de \emph{nem} a folytonos differenciálhatóságot, vagy a derivált alakját). Ez felhasználható utána az implicitfüggvény létezésének és folytonosságának belátásához az ImpFT bizonyítása során. Ezután a gyengített InvFT-től ``függetlenül'' belátható az implicitfüggvény folytonos differenciálhatósága és deriváltjának alakja is. Majd a teljes ImpFT segítségével, utána visszajöhetünk a gyengített InvFT bizonyításunkhoz, és beláthatjuk $C^1$-et, valamint a derivált alakját (ahogy mi is tettük a fenti bizonyításban).
