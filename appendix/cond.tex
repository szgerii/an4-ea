\section{Segítség az elsőrendű szükséges feltétel bizonyításához}\label{sec:cond-expl}

Tennék egy gyors, informális (értsd: lineáris algebra fogalmakkal viszonylag szabadon dobálózó) kitérőt itt az értelmezés segítéséhez. Akik jó intuícióval rendelkeznek lineáris algebra terén, ez valószínűleg minimum felesleges, de akár összezavaró is lehet (nem értek a lineáris algebrához, lehet hülyeséget mondok).

A $j_k$ oszlopok $m$ darab lineárisan független $\R^m$-beli vektort reprezentálnak, így bázist alkotnak $\R^m$-ben (és spec. a mátrix képterén is). Ez azt jelenti, hogy a maradék $n-m$ darab, $i_k$-val indexelt oszlop (ezek szintén $\R^m$-beli vektorok) egyenként előállítható a $j_k$ oszlopok lineáris kombinációjaként. Ezért is mozgunk az implicitfüggvény-tétel felé, mivel érezzük, hogy itt a ``szabadon választható'' információ kifejezéséhez nem kell mind az $n$ oszlop.

Kicsit viszont logikailag fordítottnak tűnhet (legalábbis nekem elsőre annak tűnt), amit csinálunk. Az $m$ lineárisan független változót akarjuk kifejezni a maradék $n - m$ függvényeként. Azaz azokból lesz a második (elsőtől függő) változónk az IFT-hez, amikből generálni tudjuk pont azokat, akikből ki akarjuk fejezni őket. Ez azért van, mert a lineáris közelítés (Jacobi-mátrix) oszlopaiban található lineáris összefüggések nem lesznek teljesen megfeleltethetőek a $g = 0$ feltétel változói között fellelhető feltételeknek.

Gondoljuk meg a (távolinak tűnő) matalapok tudásunkra hivatkozva egyenletrendszerek szintjén. A $g(x) = 0$ feltétel pontosan $m$ darab egyenletet fog adni, $n$ darab változóra. Mivel az $m$ egyenletből álló rendszerünk $n$ változóból áll, és $m < n$, ezért $n - m$ darab szabad változónk lesz. Továbbá a $g(x) = 0$ feltételt szeretnénk teljesíteni mindenképp, ezért akárhogy mozdulunk ki az $n - m$ szabad változónk szerint, a többi $m$-et hozzá kell tudnunk igazítani valamilyen módon, hogy $g(x) = 0$ ne sérüljön. Pont ezt a módot próbáljuk megtalálni az IFT használatával.

Tehát itt az $m$ és $n - m$ változók közötti függőség teljesen mást ír le, mint a lineáris összefüggésük, azonban az egész megoldás gondolata és intuíciója pont abból az ötletből ered, amit a lineáris algebra megmutat nekünk a $g'(c)$ mátrixból a rangfeltétel segítségével. A tényleges összefüggést pedig majd az IFT segítségével tudjuk kifejezni $g$-ből (innen fogjuk majd megkapni a Lagrange-multiplikátorokat a végleges tételhez, ha $f$-et is behozzuk a játékba).

Röviden: a lineáris algebra megadja nekünk hol keressük a függőségeket, az IFT pedig megadja ezeknek a tényleges alakját

% Geometriailag ez elképzelhető úgy, mintha egy $n-m$ dimenziós alakzatot szeretnénk kifejezni
% TODO: n = 3, m = 1 eset: bázis ``beállításra''