\chapter{*Lineáris differenciálegyenletek I.}

\setcounter{footnote}{0}

\section{Követelmények}
\begin{enumerate}
    \item Lineáris differenciálegyenletek
    \item Állandók variálásának a módszere
    \item Radioaktív bomlás felezési idejének meghatározása
\end{enumerate}

\section{Lineáris differenciálegyenletek és homogenitásuk}

\subsection{Definíció}

Vegyük a következő differenciálegyenlet esetet:

Legyen $n := 1, I \subset \R$ nyílt, $\Omega := \R$, $g,h : I \to \R$, amikre $g,h \in C$. Továbbá $f : I \times \R \to \R$ és $f(x,y) := g(x)y + h(x) \quad (x \in I, y \in \R)$.

Ekkor egy olyan $\varphi : I \to \R$ megoldást keresünk, amire $\dom_\varphi$ nyílt, $\varphi \in D$ és 
\begin{equation}\label{eq:lin-de-cond}
    \varphi'(x) = g(x)\varphi(x) + h(x) \qquad (x \in \dom_\varphi)
\end{equation}

Ezt a feladatot \textbf{lineáris differenciálegyenletnek} nevezzük.\footnote{Szokás néha csak a \eqref{eq:lin-de-cond} egyenlőséget magában is így hívni.}

Adott $\tau \in I,\ \xi \in \R$-re a $\tau \in \dom_\varphi,\ \varphi(\tau) = \xi$ feltételekkel kiegészítve pedig \emph{lineáris differenciálegyenletre vonatkozó kezdetiérték-problémának}.

\subsection{Homogén lineáris differenciálegyenlet}\label{sec:hom-lin-de}

Egy lineáris differenciálegyenlet \textbf{homogén}, ha $h \equiv 0$, illetve \textbf{inhomogén}, ha $h \not \equiv 0$.

Jelölje egy tetszőleges lineáris differenciálegyenlethez megoldáshalmazát $\M$, a hozzárendelhető homogén lin. d.e. megoldáshalmazát pedig $\M_0$ vagy $\M_h$.\footnote{
Előadáson az $\M_0$ jelölés szerepelt, a könyv és segédanyag viszont $\M_h$-t használnak. A jegyzet mindenhol az $\M_0$ jelölést fogja alkalmazni.
} Tehát,
\begin{gather}
\M := \{\varphi : I \to \R \mid \varphi \in D \land \varphi' = g\cdot \varphi + h\} \notag \\
\M_0 := \M_h := \{\varphi : I \to \R \mid \varphi \in D \land \varphi' = g\cdot \varphi\}\label{eq:lin-m0-def}
\end{gather}

\bigskip

Legyen $G : I \to \R,\, G \in D$ és $G' = g$, továbbá $\varphi_0(x) := e^{G(x)} \enspace (x \in I)$.
Ekkor $\varphi_0 \in D$ és
$$\varphi_0'(x) = G'(x)\cdot e^{G(x)} = g(x) \cdot \varphi_0(x)$$

Tehát $\varphi_0 \in \M_0$, mivel eleget tesz a \eqref{eq:lin-m0-def} feltételeinek.

\begin{theorem}[Homogén megoldások szerkezete]\label{sec:hom-mo}
\

Az $\M_0$ halmaz előállítható $\varphi_0$-ból a következő módon:
    $$\M_0 = \{c \cdot \varphi_0 : c \in \R\}$$
\end{theorem}

\begin{proof}
\

Mivel $\varphi_0 \in D$, ezért tetszőleges $c \in \R$-re $(c \cdot \varphi_0) \in D$, valamint
$$(c\cdot\varphi_0)' = c\cdot \varphi_0' = c\cdot g \cdot \varphi_0 = g\cdot (c \cdot \varphi_0)$$

Ekkor viszont definíció szerint $(c \cdot \varphi_0) \in \M_0$ (ld. \eqref{eq:lin-m0-def} egyenlőség).

Be kell még látni, hogy \emph{csak} ilyen függvények kerülhetnek $\M_0$-ba. Tegyük fel, hogy $\chi \in \M_0$.

Ekkor $\dfrac {\chi} {\varphi_0}$ differenciálható, mivel $\xi \in D, \varphi_0 \in D$ és $\varphi_0(x) \ne 0 \enspace (x \in \dom_{\varphi_0})$, ezért
$$\left(\frac \chi {\varphi_0}\right)' = \frac {\chi'\cdot \varphi_0 - \chi \cdot \varphi'_0} {\varphi_0^2} = \frac{g\cdot \chi \cdot \varphi_0 - \chi \cdot g \cdot \varphi_0}{\varphi_0^2} \equiv 0$$

Tehát mivel $\dfrac \chi {\varphi_0}$ függvény deriváltja azonosan $0$, így $\exists c \in \R : \dfrac \chi {\varphi_0} = c$.

Ekkor $\chi = c \cdot \varphi_0$, vagyis minden $\M_0$-beli tag felírható $(c \cdot \varphi_0)$ alakban valamilyen $c \in \R$-el.
\end{proof}

\section{Inhomogén eset, partikuláris megoldás}

\subsection{Állandók variálásának a módszere}

\begin{theorem}[Állandók variálásának a módszere]\label{thm:voc}
\

Bármely lineáris differenciálegyenlethez megadható olyan $m$ differenciálható függvény, amivel az $m \cdot \varphi_0$ függvény megoldás lesz, azaz
$$\exists m : I \to \R : m \in D \quad \text{és} \quad \psi := m \cdot \varphi_0 = m \cdot e^{G(x)} \in \M$$
Ebből az állításból persze az is következik, hogy $\M \ne \emptyset$.
\end{theorem}

\begin{proof}
\

Visszafelé gondolkozva fogjuk ekvivalens lépésekből megadni a létezés bizonyítását.

Az állítás, hogy $m \cdot \varphi_0 \in \M$ definíció alapján ekvivalens azzal a két állítással, hogy
\begin{equation}\label{eq:voc-cond}
m \cdot \varphi_0 \in D \quad \text{és} \quad (m \cdot \varphi_0)' = (m\cdot \varphi_0)g + h
\end{equation}
Ekkor mivel
$$(m \cdot \varphi_0)' = m'\cdot \varphi_0 + m \cdot \varphi_0' = m'\cdot \varphi_0 + m\cdot g \cdot \varphi_0$$
ezért a \eqref{eq:voc-cond} feltétel alapján
\begin{align*}
m' \cdot \varphi_0 + m \cdot g \cdot \varphi_0 &= (m \cdot \varphi_0)g + h \\
m' + m \cdot g &= m \cdot g + \frac h {\varphi_0} \\
m' &= \frac h {\varphi_0}
\end{align*}

Vagyis $m \cdot \varphi_0 \in \M$ állítással ekvivalens, hogy egyrészt $(m \cdot \varphi_0) \in D$, ami biztosan teljesül, mivel $m, \varphi_0 \in D$, másrészt a $\dfrac h {\varphi_0}$ függvénynek van primitív függvénye, ami szintén teljesül, mivel $h,\varphi_0 \in C$ és tetszőleges $x \in \dom_{\varphi_0}$-re $\varphi_0(x) \ne 0$ teljesül.
\end{proof}

\subsection{Partikuláris megoldás}

\begin{statement}
\

Egy lineáris differenciálegyenlet bármely $\varphi, \tilde\varphi : I \to \R$ megoldásaira
$$\varphi - \tilde\varphi \in \M_0$$
\end{statement}

A fenti állítás és a \ref{sec:hom-mo} tétel következménye, hogy
\begin{gather*}
\exists c \in \R : \varphi - \tilde\varphi = c\cdot \varphi_0 \implies \\
\boxed{\varphi = \tilde\varphi + c \cdot \varphi_0}
\end{gather*}

Vagyis $\tilde\varphi$ és $\varphi_0$ segítségével megadható bármely másik $\varphi$ megoldása a differenciálegyenletnek.

\emph{Tehát egyetlen elem alapján leírható az egész $\M$ halmaz.}

Az itt $\tilde\varphi$ szerepet betöltő megoldást nevezzük \textbf{partikuláris megoldásnak}.

Speciálisan a \ref{thm:voc} tételben látott $\tilde\varphi := \psi = m \cdot \varphi_0$ megoldással
\begin{equation}\label{eq:lde-psi-base}
\M = \{\psi + c \cdot \varphi_0 \mid c\in \R\}
\end{equation}

\subsection{Kezdetiérték-probléma egyértelműsége}

Minden lineáris differenciálegyenletre vonatkozó kezdetiérték-probléma \emph{egyértelműen} megoldható.

\section{Radioaktív bomlás}

\subsection{Feladat}

Képzeljük el egy (pl. radioaktív) anyag bomlását. A bomlási sebesség egyenesen arányos a még fel nem bomlott anyag mennyiségével.
\emph{A bomlás kezdetétől nézve mikor ``feleződik meg'' az anyag (azaz bomlik el a fele)?}

Adott tehát egy $m \in \R \to \R$ tömeg-idő függvény egy $\R \ni m_0 \ne 0$ kezdeti tömeggel ($m(0) = m_0$), és tegyük fel, hogy $m \in D$. Továbbá a feladat alapján ekkor
\begin{equation}\label{eq:rad-boml-de}
m'(x) = -\alpha \cdot m(x) \qquad (\R \ni \alpha > 0,\ x \in \dom_m)
\end{equation}

\subsection{Megoldás}

Jól látható, hogy a feladat egy homogén lineáris differenciálegyenletre vonatkozó kezdetiérték-probléma, ahol $g \equiv -\alpha$ (és persze $h \equiv 0$).

Ekkor a \ref{sec:hom-lin-de} szekció szerint gondolkozva, $G(x) = -\alpha \cdot x \enspace (x \in \R)$, vagyis $\varphi_0 = e^{-\alpha x} \enspace (x \in \R)$.

Továbbá a \ref{sec:hom-mo} tétel miatt,
$$\exists c \in \R : m(x) = c \cdot e^{-\alpha x}$$

Valamint a kezdetiérték-feltétel miatt:

\begin{align*}
    m(0) &= c \cdot e^{-\alpha \cdot 0} =c = m_0 \implies \\
    m(x) &= m_0 \cdot e^{-\alpha x} \quad (x \in \dom_m)
\end{align*}

Keressük ekkor azt a $T \in \dom_m$-t, amire $m(T) = \dfrac {m_0} 2$, vagyis
\begin{gather*}
m(T) = \frac {m_0} 2 = m_0 \cdot e^{-\alpha T} \implies e^{-\alpha T} =  \frac 12 
\implies \boxed {T = \frac {\ln 2} \alpha}
\end{gather*}