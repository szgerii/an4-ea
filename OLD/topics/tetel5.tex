\chapter{*Egzisztencia és unicitás}

\setcounter{footnote}{0}

\section{Követelmények}
\begin{enumerate}
    \item Lipschitz-feltétel
    \item Picard-Lindelöf-féle egzisztencia-tétel (fixpont-tétel alkalmazása)
    \item Kezdetiérték-probléma megoldásának egyértelműsége
    \item Unicitási tétel (bizonyítás nélkül)
\end{enumerate}

\section{Picard-Lindelöf-féle egzisztencia tétel}

\subsection{Kezdetiérték-probléma (emlékeztető)}

A tétel során használt jelölések gyors megkeresése érdekében itt újra megtalálható a \emph{d.e.}-k és \emph{k.é.p.}-ek definíciója:

Legyen $1 \le n \in \N$ mellett $I \subset \R$ nyílt intervallum és $\emptyset \ne \Omega \subset \R^n$ szintén egy nyílt intervallum\footnote{Ínyenceknek tartomány.}. Vegyünk továbbá egy $f : I \times \Omega \to \R^n$ függvényt, amire $f \in C$.
Adott továbbá $\tau \in I$ és $\xi \in \Omega$.

Keresünk olyan $\varphi \in I \to \Omega$ függvényt, amire
\begin{enumerate}[itemsep=1pt, topsep=1pt]
    \item $\dom_\varphi$ nyílt intervallum
    \item $\varphi \in D$ és $\varphi'(x) = f(x,\varphi(x))$
    \item $\varphi(\tau) = \xi$
\end{enumerate}

\subsection{Lipschitz-feltétel}

Egy $f$ függvény teljesíti a \textbf{Lipschitz-feltételt} (vagy \textbf{Lipschitz-folytonos}), ha
$$\exists L \in \R, L\ge 0 : \forall x,y \in \dom_f : \norm{f(x) - f(y)} \le L \cdot \norm{x-y}$$

Differenciálegyenletek kontextusában a következő formában\footnote{Ez az előadáson és a könyvben bevezetett alak.} definiáljuk:
$$\forall Q \subset \R, Q \text{ kompakt} : \exists L \in \R, L \ge 0 : \norm{f(u,v) - f(u,z)} \le L \cdot \norm{v - z} \quad (u \in I,\ v,z \in Q)$$
Ekkor az $f$ függvény (a differenciálegyenlet jobb oldala) teljesíti a \emph{Lipschitz-feltételt}. A fenti definíciókban az $L$ számot \textbf{Lipschitz-konstansnak} nevezzük.

\subsection{Differenciálegyenletek integrálegyenletként}

